

\begin{frame} %PLB
	\titre{Introduction}
	

\begin{description}[labelindent=6pt,style=multiline,leftmargin=1.3in]
	 \setlength\itemsep{1em}
	\pause
	 	\item[(Pré)nom] \textcolor{white}{blabla} \pause
	 	\item[Filière] \textcolor{white}{blabla} \pause
	 	
	 	\only<4-7>{\item[Maths] Amour ? Haine ? Une valeur intermédiaire ?}\pause
	 	\only<5-7>{\item[Prog] Quel(s) langage(s) ? }\pause
		\only<6-7>{\item[Ce cours]  Pourquoi l'avoir pris ? Vous en attendez quoi ?}
 
	 	\only<1-3>{\item[] 
	 	\item[] 
	 	\item [] \textcolor{white}{blablablablub blablablablub spozklublubloblobloble}}
	 	
	 		 	\only<4-4>{	\item[] 
	 	\item [] \textcolor{white}{blablablablub blablablablub spozklublubloblobloble}}
	 	
	 		 	\only<5-5>{\item [] \textcolor{white}{blablablablub blablablablub spozklublubloblobloble}}
	 	
\pause	 	
	 	
	 	\only<8->{\item[] 
	 	\item[] 
	 	\item [] \textcolor{white}{blablablablub blablablablub spozklublubloblobloble}}
	 	\item[Mail] \href{mailto:pbegay@ens-cachan.fr}{pbegay@ens-cachan.fr} \pause
	 	\item[] Merci de m'envoyer ça !
	\end{description}
\end{frame}

\begin{frame}
	\titre{Quelques points pratiques}
\begin{description}[labelindent=6pt,style=multiline,leftmargin=1.3in]
	 \setlength\itemsep{1em}
\pause
	 	\item[Présence] non obligatoire \pause à vos risques et périls ! \pause
	 	\item[] Pas de poly (mais \textit{slides} très verbeuses)\pause
	 \item[Retards] silencieux \pause
	 \item[Moodle] jamais (donc écrivez-moi vraiment !)
	 \pause
	 \item[Note] Partiel (mi-nov.) + Exam (janvier) \pause
	 \item[] 5 DMs \pause optionnels
	\end{description}
	
\end{frame}


\begin{frame} %PLB
	\titre{Bibliographie}
	

\begin{description}[labelindent=6pt,style=multiline,leftmargin=1.3in]
	 \setlength\itemsep{1em}
\item[Le classique] Logic, Language, and Meaning, Volume 1, `le Gamut`, chapitres 1 à 4
\item[] pdf trouvable en ligne (par exemple ici : \url{http://cpc.cx/mzl})\pause
\item[Bonus] Bibliographie \textit{classique} plus complète ici : \url{http://cpc.cx/mzm}\pause
\item[Histoire] Logicomix (roman graphique, Doxiadis, Papadimitriou, Papadatos \& Donna)\pause
\item[Culture scientifique] L'intelligence artificielle (BD, Lafargue \& Montaigne)

	\end{description}
\end{frame}



\begin{frame}
\titre{Première énigme}

\begin{itemize}
\item Par un retour de karma attendu de longue date, le colonel Moutarde s'est fait tuer cette nuit. 

\item Seules 3 personnes auraient pu commettre le meurtre : le sergent Garcia, Vald et Ronald McDonalds.
\end{itemize}
\end{frame}




\begin{frame}
%\titre{Première énigme}
\titre{Les faits}

\begin{itemize}

	\item[\textcolor{blue}1] Le cadavre du colonel a été retrouvé dans sa cuisine \pause

    \item[\textcolor{blue}2] Jean-Michel a fêté son anniversaire dans le restaurant de Ronald la nuit du meurtre\pause

	\item[\textcolor{blue}3] Vald a rendu visite au sergent Garcia et Ronald McDonalds deux jours avant le meurtre\pause

     \item[\textcolor{blue}4] Tous les employés de Ronald sont malades \pause

     \item[\textcolor{blue}5] Le sergent Garcia est champion de Jiu-jitsu brésilien\pause
 
	\item[\textcolor{blue}6] Le sergent Garcia et Ronald McDonalds sont constamment menottés l'un à l'autre\pause
	
	\item[\textcolor{blue}7] Le colonel était allergique aux big macs de Ronald\pause

\end{itemize}

$\Rightarrow$ Lequel des suspects a tué le colonel moutarde et pourquoi ?

\end{frame}


\begin{frame}
%\titre{Première énigme}
\titre{Les faits}

\begin{itemize}

     \item[\textcolor{blue}4] Tous les employés de Ronald sont malades \pause
\begin{itemize}
     \item[\textcolor{orange}{4B}] Les employés de Ronald ne peuvent pas travailler\pause
     
     \item[\textcolor{orange}{4T}] Si quelqu'un travaille au restaurant, c'est Ronald\pause
     
\end{itemize}
     \item[\textcolor{blue}2] Jean-Michel a fêté son anniversaire dans le restaurant de Ronald la nuit du meurtre\pause 
     \begin{itemize}
     \item[\textcolor{orange}{2B}] Quelqu'un a travaillé au restaurant la nuit du meurtre\pause
     \end{itemize}
     \item[\textcolor{red}8] \textcolor{orange}{4T} + \textcolor{orange}{2B} $\Rightarrow$ Ronald a travaillé au restaurant la nuit du meurtre\pause 
\begin{itemize}
     \item[\textcolor{orange}{8B}] Ronald n'était pas sur les lieux du crime la nuit du meurtre
     \end{itemize}
     \pause
     
     \item[\textcolor{red}{9}] \textcolor{blue}{6} (menottes) + \textcolor{orange}{8B} $\Rightarrow$ Garcia n'était pas sur les lieux du crime la nuit du meurtre

\end{itemize}
\end{frame}



\begin{frame}
%\titre{Première énigme}
\titre{Les faits}

\begin{itemize}

   
     \item[\textcolor{orange}{8B}] Ronald n'était pas sur les lieux du crime la nuit du meurtre\pause
     \begin{itemize}
     \item[\textcolor{orange}{8T}] Ronald n'a pas commis le meurtre\pause
     \end{itemize}
     \item[\textcolor{red}{9}] Garcia n'était pas sur les lieux du crime la nuit du meurtre

     \begin{itemize}
     \item[\textcolor{orange}{9B}] Garcia n'a pas commis le meurtre\pause
     \end{itemize}

     \item[\textcolor{blue}0] Seules 3 personnes auraient pu commettre le meurtre\pause
     
     \begin{itemize}
     \item[\textcolor{orange}{0B}] Ronald a commis le meurtre ou Garcia a commis le meurtre ou Vald a commis le meurtre\pause
     \end{itemize}

     \item[Concl.] \textcolor{orange}{8T} + \textcolor{orange}{9B} + \textcolor{orange}{0B} $\Rightarrow$ Vald a commis le meurtre\pause
     
     \item Des remarques ? \pause Est-ce \textbf{logique} ?

\end{itemize}
\end{frame}





\begin{frame}
	\titre{Notions de base}
	 Logique ?\pause\newline
	 \textcolor{white}{lol}
	\begin{description}[labelindent=6pt,style=multiline,leftmargin=1.3in]
		 \setlength\itemsep{1.4em}
		 \item[Larousse] Manière dont les faits s'enchaînent, découlent les uns des autres\pause
		\item[Larousse bis] Science du raisonnement en lui-même, abstraction faite de la matière à laquelle il s'applique et de tout processus psychologique\pause
		
		%\item[Larousse ter] Étude des \colorAt{8}{red}{automates}, des \colorAt{8}{red}{automatismes}, et de leurs composants et circuits électroniques correspondants\pause

		\item[Wikipedia] L'étude des règles formelles que doit respecter toute argumentation correcte

	\end{description}


\end{frame}
 	
\begin{frame}
	\titre{Notions de base}
	\begin{description}[labelindent=6pt,style=multiline,leftmargin=1.3in]
		 \setlength\itemsep{1.4em}
	  \item[Larousse] Manière dont les faits s'enchaînent, \colorAt{2}{red}{découlent les uns des autres}\pause\pause
	\end{description}

\begin{itemize}
\item[] \textcolor{white}{lol}
\item On a combiné les informations \textit{de base} pour en obtenir de nouvelles \pause qui ont elles-mêmes étaient combinées\pause
\item[] \textcolor{white}{lol}
\item $\approx$ Lego\pause
\item[] \textcolor{white}{lol}
\item Analogie avec le \textbf{calcul}
\end{itemize}

\end{frame}



\begin{frame}
	\titre{Notions de base}
	\begin{description}[labelindent=6pt,style=multiline,leftmargin=1.3in]
		 \setlength\itemsep{1.4em}
	  \item[Larousse] Science du raisonnement en lui-même, \colorAt{5}{blue}{abstraction faite de la matière à laquelle il s'applique} et \colorAt{2}{red}{de tout processus psychologique}\pause\pause
	\end{description}

\begin{itemize}
\item[] \textcolor{white}{lol}
\item La logique \textit{transcende} son \textit{application} chez les humains\pause
\item[] \textcolor{white}{lol} 	
\item Savoir quelle région du cerveau est activée par telle énigme ou quelles cellules font tel truc relève d'autres domaines (resp. psycho et neurologie)\pause\pause
\item[] \textcolor{white}{lol}
\item On a des \textit{patterns}
\end{itemize}

\end{frame}
 	


\begin{frame}
	\titre{Notions de base}

\begin{itemize}

\item[] Rappel : 

     \item[\textcolor{orange}{0B}] Ronald a commis le meurtre ou Garcia a commis le meurtre ou Vald a commis le meurtre

        \item[\textcolor{orange}{8T}] Ronald n'a pas commis le meurtre
     \item[\textcolor{orange}{9B}] Garcia n'a pas commis le meurtre
\pause
     \item[Concl.] \textcolor{orange}{8T} + \textcolor{orange}{9B} + \textcolor{orange}{0B} $\Rightarrow$ Vald a commis le meurtre\pause
     
     \item[] \textcolor{white}{lol}
     \item Sans doute que ça marche avec d'autres personnages ou autre chose qu'un meurtre
   
\end{itemize}
\end{frame}
 	
 	
 	
 	

\begin{frame}
	\titre{Notions de base}
	\begin{description}[labelindent=6pt,style=multiline,leftmargin=1.3in]
		 \setlength\itemsep{1.4em}
	  \item[Wikipedia] L'étude des \colorAt{2}{red}{règles formelles} que doit respecter toute argumentation correcte \pause\pause
	  \item[] \textcolor{white}{lol}
	\end{description}

\begin{itemize}

\only<1-2>{
     \item[\textcolor{white}{0B}] \textcolor{white}{Ronald a commis le meurtre ou Garcia a commis le meurtre ou Vald a commis le meurtre}

        \item[\textcolor{white}{8T}] \textcolor{white}{Ronald n'a pas commis le meurtre}
     \item[\textcolor{white}{9B}] \textcolor{white}{Garcia n'a pas commis le meurtre}

     \item[] \textcolor{white}{8T} \textcolor{white}{+} \textcolor{white}{9B} \textcolor{white}{+} \textcolor{white}{0B} \textcolor{white}{$\Rightarrow$ Vald a commis le meurtre}
     }
     
     
\only<3>{
     \item[\textcolor{orange}{0B}] Ronald a commis le meurtre ou Garcia a commis le meurtre ou Vald a commis le meurtre

        \item[\textcolor{orange}{8T}] Ronald n'a pas commis le meurtre
     \item[\textcolor{orange}{9B}] Garcia n'a pas commis le meurtre

     \item[Concl.] \textcolor{orange}{8T} + \textcolor{orange}{9B} + \textcolor{orange}{0B} $\Rightarrow$ Vald a commis le meurtre\pause
     }
     
     
\only<4>{
     \item[\textcolor{orange}{0B}] Perso1 a fait X ou Perso2 a fait X ou Perso3 a fait X \textcolor{white}{ou ta mère a fait un truc}

        \item[\textcolor{orange}{8T}] Perso1 n'a pas fait X
     \item[\textcolor{orange}{9B}] Perso2 n'a pas fait X

     \item[Concl.] \textcolor{orange}{8T} + \textcolor{orange}{9B} + \textcolor{orange}{0B} $\Rightarrow$ Perso3 a fait X
     }
     
\only<5->{
     \item[\textcolor{orange}{0B}] A est vrai ou B est vrai ou C est vrai\textcolor{white}{ou ta mère a fait un tru lol ptdrc}

        \item[\textcolor{orange}{8T}] A n'est pas vrai
     \item[\textcolor{orange}{9B}] B n'est pas vrai

     \item[Concl.] \textcolor{orange}{8T} + \textcolor{orange}{9B} + \textcolor{orange}{0B} $\Rightarrow$ C est vrai
     }
      \pause
      \pause
      \pause
      \item \textcolor{white}{ou ta mère a fait un truc}
      \item On ne peut sans doute pas faire plus abstrait\pause , mais est-ce \textit{minimal} ?
     
\end{itemize}
\end{frame}
 	

%-------------
		%\item[Larousse ter] Étude des \colorAt{8}{red}{automates}, des \colorAt{8}{red}{automatismes}, et de leurs composants et circuits électroniques correspondants\pause

\begin{frame}
	\titre{Notions de base}

	\begin{description}[labelindent=6pt,style=multiline,leftmargin=1.3in]
		 \setlength\itemsep{1.4em}
	  \item[Larousse ter] Étude des \colorAt{3}{red}{automates}, des \colorAt{3}{red}{automatismes}, et de leurs composants et circuits électroniques correspondants\pause
	\end{description}

\begin{itemize}
\item[] 
\item[] Celle-ci a l'air bizarre (des circuits électroniques ??)\pause , mais elle est en fait extrêmement pertinente\pause
\item[]
\item[] On s'y attardera (peut-être) à la fin du semestre
\end{itemize}
\end{frame}
 	
 	
 	



\begin{frame}
	\titre{Notions de base}
	\only<1>{Dernière définition !}\pause
	 \textit{Logic studies the relationship between language, meaning and (proof) method} - de Moura et Bjørner\pause
	 \vspace{0.1cm}
	
	 Différencie 3 concepts:\pause
	 
	\begin{description}[labelindent=6pt,style=multiline,leftmargin=1.3in]
		 \setlength\itemsep{1.4em}

		\item[Preuve] Le raisonnement \pause
		\item[Sens] L'expression formelle des hypothèses et conclusions\pause
		\item[] Cette façon de s'exprimer forme un langage\pause
		\item[Langage] Les correspondances entre ce langage et celui du quotidien, dit naturel


	\end{description}


\end{frame}
 	
 	
 	
