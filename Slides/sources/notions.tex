



%
%\begin{frame}
%	\titre{Notions de base}
% 
%	\begin{description}[labelindent=6pt,style=multiline,leftmargin=1.3in]
%		 \setlength\itemsep{1.4em}
%
%		\item[Raisonnement] Le raisonnement \pause
%		\item[Sens] L'expression formelle des hypothèses et conclusions\pause
%		\item[] Cette façon de s'exprimer forme un langage\pause
%		\item[Langage] Les correspondances entre ce langage et celui du quotidien, dit naturel
%
%
%	\end{description}
%
%
%\end{frame}

%
%\begin{frame}
%	\titre{Notions de base}
%	
%	\begin{description}[labelindent=6pt,style=multiline,leftmargin=1.3in]
%		 \setlength\itemsep{1.4em}
%
%		\item[Logique] \only<1>{?} \only<2>{Etude du raisonnement ?} \only<3->{Etude des \textbf{inférences}} \pause \pause
%\pause
%		\item[Inférence]  Dériver une \textbf{conclusion} à partir de \textbf{prémisses} vraies \pause ou supposées vraies \pause
%
%		\item[] Processus naturel et psychologique \pause
%		\item[] mais pas l'objet d'étude des logiciens (et linguistes) \pause
%		\item[Plutôt] la façon dont \textbf{s'expriment} les raisonnements \pause : on va s'intéresser à leur \textbf{forme}.
%
%
%	\end{description}
%
%
%\end{frame}

%
%\begin{frame}
%	\titre{Notions de base}
%	
%	\begin{description}[labelindent=6pt,style=multiline,leftmargin=1.3in]
%		 \setlength\itemsep{1.4em}
%
%		\item[Logique] Etude des propriétés formelles des inférences correctes \pause
%		\item[]
%
%	\end{description}
%
%
%\end{frame}


\begin{frame}
	\titre{Notions de base : inférence}
	
	\begin{description}[labelindent=6pt,style=multiline,leftmargin=1.3in]
		 \setlength\itemsep{1.4em}	

		\item[définition] Dérivation une \textbf{conclusion} à partir de \textbf{prémisses} vraies \pause ou supposées vraies \pause

	\item[Remarque] Un raisonnement est une ou plusieurs inférences imbriquées
	
	\end{description}
\end{frame}


\begin{frame}
	\titre{Notions de base : syllogisme}
	
	\begin{description}[labelindent=6pt,style=multiline,leftmargin=1.3in]
		 \setlength\itemsep{1.4em}


		\item[définition] Mise en forme d'une inférence\pause

	\item[Exemples] 
\begin{tabular}{l}
Je pense\\\hline Je suis\\ 
\end{tabular}

\pause
\item[] \begin{tabular}{l}
Tous les canards boitent\\José est un canard\\\hline José boite\\ 
\end{tabular}

\pause
%\item[] \begin{tabular}{l}
%Tous les hommes sont mortels\\Les mortes sont désespérés\\Les désespérés font des bêtises\\Rouler sans permis est une bêtise\\\hline Il y a des hommes qui roulent sans permis\\ 
%\end{tabular}
\item[] \begin{tabular}{l}
Nulle chaise ne respire\\Tout Homme respire\\\hline Aucun Homme n'est une chaise\\ 
\end{tabular}

	\end{description}

\end{frame}



\begin{frame}
	\titre{Notions de base : syllogistique}
	\only<1>{Etude des syllogismes}
%\pause
	%\only<2>{\includegraphics[scale=0.083]{silog.JPG}}
	
\end{frame}

\begin{frame}
	\titre{Notions de base : application}
	
\begin{tabular}{lr}
Tous les canards boitent & \hspace{2cm} prémisse numéro 1 \\
José est un canard & prémisse numéro 2 \\ \cline{1-1}
 José boite & conclusion\\ \\ \\
\end{tabular}

\pause

Ce syllogisme vous semble-t'il \textit{raisonnable} ? \pause \newline 

Normalement oui \pause : on va du général au spécifique% en \textit{appliquant} la première prémisse avec José, ce qu'on a le droit de faire grâce à la deuxième \newline \pause

%Analogie avec, par exemple, $f(x) = 3\times x + 2$, donc $f(5) = 17$

%
%\begin{description}[labelindent=6pt,style=multiline,leftmargin=1.3in]
%		 \setlength\itemsep{1.4em}
%
%
%		\item[Attention] On ne met pas n'importe quoi en prémisse ou conclusion \pause
%		\item[] On met des \textbf{propositions} \pause
%		\item[] Une expression qu'on peut considérer comme \textbf{vraie} ou \textbf{fausse}
%
%
%	\end{description}

\end{frame}



\begin{frame}
	\titre{Notions de base : application bis}
	
\begin{tabular}{lr}
Tous les canards boitent & \hspace{2cm} prémisse numéro 1 \\ \cline{1-1}
 Jean-Michel boite & conclusion\\ \\ \\
\end{tabular}

\pause

Ce syllogisme vous semble-t'il \textit{raisonnable} ? \pause \newline 

Normalement non \pause : la prémisse 1 est un \textit{contrat} : si tu me \textit{garantis} qu'un objet $x$ est un \textit{canard}, je te garantis qu'il boite.\newline \pause

Il nous manque ici l'information, assertée par une prémisse, que Jean-Michel est un canard \pause \newline

%C'est comme avoir $f(x) = 3\times x + 2$ et essayer de calculer $f(patate)$ \pause \newline

Analogie avec les types en programmation ($int \rightarrow int$)

\end{frame}


\begin{frame}
	\titre{Notions de base : exclusivité}
	%TODO : rajouter une slide pour discuter de ce que recouvre ou non `Homme'
	%ie. vivant ou non (définition précise des termes, tout ça tout ça)
\begin{tabular}{lr}
Aucune chaise ne respire& \hspace{1cm} prémisse numéro 1 \\ 
Tout être humain respire & prémisse numéro 2 \\ \cline{1-1}
Aucun être humain n'est une chaise & conclusion\\ \\ \\
\end{tabular}

\pause

Ce syllogisme vous semble-t'il \textit{raisonnable} ? \pause \newline 

Normalement oui \pause : la prémisse 1 dit qu'être une chaise et respirer sont deux propriétés incompatibles (ou mutuellement exclusives)\newline \pause

La prémisse 2 dit que les Hommes respirent, et donc qu'ils ont `fait leur choix` entre les deux propriétés

\end{frame}


\begin{frame}
	\titre{Notions de base : validité vs. vérité}
	
\begin{tabular}{lr}
Aucun enfant ne respire& \hspace{1cm} prémisse numéro 1 \\ 
Tout être humain respire & prémisse numéro 2 \\ \cline{1-1}
Aucun être humain n'est un enfant& conclusion\\ \\ \\
\end{tabular}

\pause

Ce syllogisme vous semble-t'il \only<2-6>{\textit{raisonnable}} \only<7>{\textbf{valide}} ? \pause \newline 

Normalement oui \pause : au niveau du raisonnement, c'est complètement équivalent (ou isomorphe) à l'exemple précédent \newline \pause

On suppose \textbf{toujours} les prémisses vraies\newline \pause

`Dans un monde où toutes les prémisses sont vraies, puis-je affirmer de façon raisonnable la conclusion ? ` \pause

\end{frame}


\begin{frame}
	\titre{Notions de base : rigueur}
	
\begin{tabular}{lr}
L'immense majorité des canards boite & \hspace{1cm} prémisse numéro 1 \\ 
Jean-Claude est un canard & prémisse numéro 2 \\ \cline{1-1}
Jean-Claude boite & conclusion\\ \\ \\
\end{tabular}

\pause

Ce syllogisme vous semble-t'il \textbf{valide} ? \pause \newline 

Normalement non \pause : on doit toujours \textit{chercher la petite bête}\newline \pause

`Dans un monde où l'ensemble des prémisses est vrai, puis-je affirmer \only<5>{de façon raisonnable} \only<6->{\textbf{de façon \underline{certaine}}} la conclusion ? ` \newline \pause \pause 

La logique, c'est (aussi) l'art d'être chiant

\end{frame}


\begin{frame}
	\titre{Notions de base : abstraction}
	
\begin{tabular}{lr}
Je pense & \hspace{2cm} prémisse numéro 1 \\ \cline{1-1}
 Je suis & conclusion\\ \\ \\
\end{tabular}

\pause

Ce syllogisme vous semble-t'il \textbf{valide} ? \pause \newline 

Normalement non \pause : aucun lien formel entre la prémisse et la conclusion\newline \pause

On va chercher un système \textbf{abstrait} \pause et \textbf{minimal} (voire fini)  \newline \pause

\only<1-6>{\begin{tabular}{l}
\textcolor{white}{Tous les canards boitent}\\\textcolor{white}{José est un canard}\\\textcolor{white}{José boite}\\ 
\end{tabular}}
\only<7>{\begin{tabular}{l}
Tous les canards boitent\\José est un canard\\\hline José boite\\ 
\end{tabular}}
\pause
\only<8>{\begin{tabular}{l}
Tous les canards ont la propriété de boiter\\José est un canard\\\hline José a la propriété de boiter\\ 
\end{tabular}}
\pause
\only<9>{\begin{tabular}{l}
Tous les X ont la propriété Y\\
Z est un X\\ \cline{1-1}
Z a la propriété Y\\
\end{tabular}}
\end{frame}

% ajouts sur l'abstraction
% -----------------------------------


\begin{frame}
	\titre{Notions de base : abstraction}
	
Cette recherche d'abstraction est le prolongement naturel de la distinction qu'on a faite entre \textbf{vérité} (dont on se fiche) et \textbf{validité} (qu'on cherche à \textit{formaliser}, c'est à dire qu'on veut décrire intégralement à l'aide d'un ensemble fini de règles) \pause \newline 

L'idée, c'est d'avoir des principes \textbf{généraux}, qui marcheront même dans le plus bizarre des mondes, puis de les appliquer à (ou instancier avec) un monde en particulier qui contiendra plein de règles comme \newline `ne pas valider LL $\rightarrow$ ne pas valider son année` \pause \newline 

Cette phrase n'est pas très chouette, peut-on la reformuler de façon \textbf{équivalente} mais moins négative ?

\end{frame}

\begin{frame}
	\titre{Notions de base : abstraction}
	
L'idée que les \textit{règles particulières} de \textbf{notre} monde sont à distinguer de certaines règles plus fondamentales, ou universelles, préfigure le \textbf{générativisme}. \newline \pause

La linguistique générative est une théorie portée par Noam Chomsky (et ses camarades) à partir des années 50 qui postule une structure commune à toutes les langues. \newline \pause

Intuitivement, il existe (selon eux) une langue abstraite qui contient par exemple la notion de sujet / verbe / complément qui est ensuite \textit{instanciée} en français, anglais, coréen, turque etc... avec à chaque fois différents paramètres (vocabulaire, ordre SVO etc...)


\end{frame}


\begin{frame}
	\titre{Notions de base : abstraction}
	\pause
Une analogie expérimentale : \pause Fortnite vs. PUBG \newline \pause
	
Les deux jeux, ainsi que les $1329$ autres du genre, diffèrent dans leurs directions artistiques, armes, \textit{maps} etc\pause, mais ces différences sont transcendées par un ADN de base, ou le genre donc, c'est à dire les grands principes (Une île, 100 clampins, une zone qui se réduit, etc) \newline \pause

On peut séparer les `grandes règles`, qui distinguent le \textit{Battle royale} d'autres genre de jeux, des spécificités de chaque BR qui les distinguent les uns des autres. \newline \pause

Même différence entre l'étude des \textbf{langues} \pause et du \textbf{langage}

\end{frame}



\begin{frame}
	\titre{Notions de base : abstraction}

En effet, la communication repose sur une longue \textit{chaine de production}, qui part de l'idée abstraite qu'on veut exprimer et débouche sur une suite de sons (si on est à l'oral).\pause \newline

En gros, on construit d'abord le \textit{sens} précis de l'idée (c'est la \textbf{sémantique}), on traduit ce sens en structure de phrase (c'est la \textbf{syntaxe}), on calcule la suite de sons qui correspond à cette phrase (c'est la \textbf{phonologie}) et on effectue les mouvements articulatoires correspondant (c'est la \textbf{phonétique}).\newline


\end{frame}

\begin{frame}
	\titre{Notions de base : abstraction}

C'est une vision extrêmement simplifiée, mais qui permet d'entrevoir la diversité des processus en jeu pour simplement parler.\pause \newline

Chaque processus étant en soi extrêmement complexe, il est raisonnable de les étudier séparément. Par exemple, le raisonnement fait partie du \textit{sens} \pause(même si on peut l'observer - notamment - via la syntaxe !).\pause \newline

Clairement, la syntaxe et (surtout) la phonétique et la phonologie dépendent de la langue utilisée, mais est-ce le cas de la sémantique ?


\end{frame}



\begin{frame}
	\titre{Notions de base : abstraction}

Ca rejoint l'Hypothèse de Sapir-Whorf\footnote{dont vous avez peut-être entendu parler dans l'excellent film `Premier contact` / `\textit{Arrival}`}, selon laquelle notre vision du monde dépend directement de notre langue. \pause \newline

La question n'est évidemment pas résolue (ni très clairement posée). \pause \newline

Voir cependant le cas très particulier du Pirahã, et de sa relation avec le générativisme et Sapir-Whorf (\textcolor{blue}{\href{http://cpc.cx/khG}{ici}} pour commencer).

\end{frame}

\begin{frame}
	\titre{Notions de base : abstraction}
\only<1->{\textcolor{white}{Olala ligne cachée} \newline}
BREF \pause \newline
	
En syllogistique, tous les mots pas \textit{fonctionnels} (tout ce qui n'est pas `tout`, `aucun`, `est`, `ne pas` etc) sont à considérer comme des variables. \newline 


\only<1-2>{
\textcolor{white}{\begin{tabular}{l}
Tout ce qui est rare est cher\\José arrose la route\\\hline La route est mouillée\\ 
\end{tabular}}
\textcolor{white}{discret saut à la ligne}\newline

\textcolor{white}{Ca, c'est ok du point de vue de la syllogistique parce qu'on considère `être bon marché` et `être rare` comme deux propriétés lambdas}
}

\pause
\only<3>{\begin{tabular}{l}
\textcolor{white}{Tout ce qui est rare est cher}\\\textcolor{black}{José arrose la route}\\\hline \textcolor{black}{La route est mouillée}\\ 
\end{tabular}
\textcolor{white}{discret saut à la ligne}\newline

\textcolor{white}{Ca, c'est ok du point de vue de la syllogistique parce qu'on considère `être bon marché` et `être rare` comme deux propriétés lambdas}
}

\pause
\only<4>{\begin{tabular}{l}
\textcolor{white}{Tout ce qui est rare est cher}\\X effectue l'action Y sur Z\\\hline Z est R\\ 
\end{tabular}
\textcolor{white}{discret saut à la ligne} \newline

\textcolor{white}{Ca, c'est ok du point de vue de la syllogistique parce qu'on considère `être bon marché` et `être rare` comme deux propriétés lambdas}
}


\pause
\only<5->{
\begin{tabular}{l}
Tout ce qui est rare est cher\\Un cheval bon marché est rare\\\hline Un cheval bon marché est cher\\ 
\end{tabular}
\pause
\textcolor{white}{discret saut à la ligne} \newline

Ca, c'est ok du point de vue de la syllogistique parce qu'on considère `être bon marché` et `être rare` comme deux propriétés lambdas
}

\end{frame}
