
\section{Déduction naturelle}

\begin{frame}
	\titre{Déduction naturelle}
	
	 Les tables de vérité, c'est une méthode de preuve solide, mais un peu bourrine et laborieuse\pause\newline
	
	Mais, plus gênant, les tables de vérité ne \textit{racontent pas d'histoire} : \pause le raisonnement - et sa construction - n'apparaissent pas. \pause Ce sont finalement plus des faits que des preuves\pause\newline
	 
	Une logique est livrée avec ses \textbf{systèmes déductifs}, cad des formalismes décrivant la construction de preuves\pause\newline
	
	En logique prop., les plus canoniques sont \textbf{La déduction à la Hilbert}, \textbf{Le calcul des séquents} et \textbf{La déduction naturelle}. C'est à ce dernier qu'on va s'intéresser
	
\end{frame}

%----------------------------------------

\begin{frame}
	\titre{Déduction naturelle}
	
	 On introduit un nouveau symbole : $\vdash$\pause\newline
	 
	 $\phi_1, \phi_2 \dots, \phi_n \vdash \psi \equiv $ `Avec les hypothèses $\phi_1, \phi_2 \dots \phi_n$, on peut déduire $\psi$`\pause\newline
	 
	 Convention : on va en général appeler un ensemble d'hypothèses $\Gamma$\pause\newline
	 
	 $\Gamma, \phi \vdash \phi$ $\equiv$ `Avec un tas d'hypothèses, dont $\phi$, on peut déduire $\phi$` \pause : c'est l'\textbf{axiome}\pause\newline
	 
	 On va aussi avoir des règles qui ressemblent à ça :
	 
	 \begin{prooftree}
\AxiomC{Prémisse 1}
%\UnaryInfC{B}
\AxiomC{Prémisse 2}
\BinaryInfC{Conclusion}
\end{prooftree}

%Vous pouvez remarquer que les prémisses sont alignées au lieu d'être empilées (comme dans Port-Royal), on va très rapidement voir pourquoi ! 	
	
\end{frame}

%----------------------------------------

\begin{frame}
	\titre{Déduction naturelle}
	
	Par exemple, la règle suivante : 
	 \begin{prooftree}
\AxiomC{$\Gamma \vdash \phi$}
%\UnaryInfC{B}
\AxiomC{$\Gamma \vdash \psi$}
\BinaryInfC{$\Gamma \vdash (\phi \wedge \psi)$}
\end{prooftree}

\pause
`Si on a une preuve $\phi$ à partir de $\Gamma$ et qu'on a une preuve de $\psi$ à partir de $\Gamma$, alors on a une preuve de ($ \phi \wedge \psi$) à partir de $\Gamma$`\pause\newline

Cette règle s'appelle l'\textbf{introduction du $\wedge$} (ou $\wedge$-introduction)\pause\newline

Vous pouvez remarquer que les prémisses sont alignées au lieu d'être empilées (comme dans Port-Royal), on va très rapidement voir pourquoi ! 	
	

%Vous pouvez remarquer que les prémisses sont alignées au lieu d'être empilées (comme dans Port-Royal), on va très rapidement voir pourquoi ! 	
	
\end{frame}

%----------------------------------------

\begin{frame}
	\titre{Déduction naturelle}
	
	Règles duales : 
	 \begin{prooftree}
\AxiomC{$\Gamma \vdash (\phi \wedge \psi)$}
%\UnaryInfC{B}
\UnaryInfC{$\Gamma \vdash \phi$}
\end{prooftree}

\begin{prooftree}
\AxiomC{$\Gamma \vdash (\phi \wedge \psi)$}
%\UnaryInfC{B}
\UnaryInfC{$\Gamma \vdash \psi$}
\end{prooftree}


\pause

`Si on a une preuve de $(\phi \wedge \psi)$ à partir de $\Gamma$, alors on a une preuve de $\phi$ (resp. $\psi$) à partir de $\Gamma$`. C'est les règles d'\textbf{élimination du $\wedge$} (ou $\wedge$-élimination)\pause\newline

Intuitivement, ces deux règles disent qu'on peut \textit{perdre de l'information} : `si je sais que machin et truc, alors \textbf{en particulier} je sais que machin (ou truc)`
	
\end{frame}

%----------------------------------------

\begin{frame}
	\titre{Déduction naturelle}
	
	Et cette règle un peu bizarre : 
	 \begin{prooftree}
\AxiomC{$\Gamma,\phi \vdash \psi$}
%\UnaryInfC{B}
\UnaryInfC{$\Gamma \vdash (\phi \rightarrow \psi)$}
\end{prooftree}

\pause
`Si on a une preuve $\psi$ à partir de $\Gamma$ et $\phi$, alors on a une preuve de ($ \phi \rightarrow \psi$) à partir de $\Gamma$`\pause\newline

Cette règle un peu absconce (c'est une nécessité technique du formalisme) s'appelle l'\textbf{introduction de la $\rightarrow$} (ou $\rightarrow$-introduction)\pause\newline

On n'a pas encore vu toutes les règles, mais ce qu'on a nous suffit déjà à faire une preuve non-triviale : $ \vdash ((\phi \wedge \psi) \rightarrow (\psi \wedge \phi))$
	

%Vous pouvez remarquer que les prémisses sont alignées au lieu d'être empilées (comme dans Port-Royal), on va très rapidement voir pourquoi ! 	
	
\end{frame}

%----------------------------------------

\begin{frame}
	%\titre{Déduction naturelle}
	
	\only<1-3>{
	 \begin{prooftree}
\AxiomC{$(\phi \wedge \psi) \vdash (\phi \wedge \psi)$}
\RightLabel{\textcolor{white}{$\wedge$-elimination}}
\UnaryInfC{$(\phi \wedge \psi) \vdash \psi$}
\AxiomC{$(\phi \wedge \psi) \vdash (\phi \wedge \psi)$}
\RightLabel{\textcolor{white}{$\wedge$-elim}}
\UnaryInfC{$(\phi \wedge \psi) \vdash \phi$}
\RightLabel{\textcolor{white}{$\wedge$-intro}}
\BinaryInfC{$(\phi \wedge \psi) \vdash (\psi \wedge \phi)$}
%\UnaryInfC{B}
\RightLabel{\textcolor{white}{$\rightarrow$-introduction}}
\UnaryInfC{$ \vdash ((\phi \wedge \psi) \rightarrow (\psi \wedge \phi))$}
\end{prooftree}}

\only<4-5>{
	 \begin{prooftree}

\AxiomC{$(\phi \wedge \psi) \vdash (\phi \wedge \psi)$}
\RightLabel{\textcolor{white}{$\wedge$-elimination}}
\UnaryInfC{$(\phi \wedge \psi) \vdash \psi$}
\AxiomC{$(\phi \wedge \psi) \vdash (\phi \wedge \psi)$}
\RightLabel{\textcolor{white}{$\wedge$-elim}}
\UnaryInfC{$(\phi \wedge \psi) \vdash \phi$}
\RightLabel{\textcolor{white}{$\wedge$-intro}}
\BinaryInfC{$(\phi \wedge \psi) \vdash (\psi \wedge \phi)$}
%\UnaryInfC{B}
\RightLabel{$\rightarrow$-introduction}
\UnaryInfC{$ \vdash ((\phi \wedge \psi) \rightarrow (\psi \wedge \phi))$}
\end{prooftree}
}


\only<6-7>{
	 \begin{prooftree}

\AxiomC{$(\phi \wedge \psi) \vdash (\phi \wedge \psi)$}
\RightLabel{\textcolor{white}{$\wedge$-elimination}}
\UnaryInfC{$(\phi \wedge \psi) \vdash \psi$}
\AxiomC{$(\phi \wedge \psi) \vdash (\phi \wedge \psi)$}
\RightLabel{\textcolor{white}{$\wedge$-elim}}
\UnaryInfC{$(\phi \wedge \psi) \vdash \phi$}
\RightLabel{$\wedge$-intro}
\BinaryInfC{$(\phi \wedge \psi) \vdash (\psi \wedge \phi)$}
%\UnaryInfC{B}
\RightLabel{$\rightarrow$-introduction}
\UnaryInfC{$ \vdash ((\phi \wedge \psi) \rightarrow (\psi \wedge \phi))$}
\end{prooftree}
}

%\RightLabel{$\wedge$-elimination}

\only<8>{
	 \begin{prooftree}

\AxiomC{$(\phi \wedge \psi) \vdash (\phi \wedge \psi)$}
\RightLabel{$\wedge$-elimination}
\UnaryInfC{$(\phi \wedge \psi) \vdash \psi$}
\AxiomC{$(\phi \wedge \psi) \vdash (\phi \wedge \psi)$}
\RightLabel{$\wedge$-elim}
\UnaryInfC{$(\phi \wedge \psi) \vdash \phi$}
\RightLabel{$\wedge$-intro}
\BinaryInfC{$(\phi \wedge \psi) \vdash (\psi \wedge \phi)$}
%\UnaryInfC{B}
\RightLabel{$\rightarrow$-introduction}
\UnaryInfC{$ \vdash ((\phi \wedge \psi) \rightarrow (\psi \wedge \phi))$}
\end{prooftree}
}
\pause
%

En lisant à partir du bas :\vspace{0.14cm}\newline\pause
On veut prouver `$(\phi \wedge \psi) \rightarrow (\psi \wedge \phi)$`. On ajoute donc $\phi \wedge \psi$ à notre ensemble d'hypothèses en utilisant la $\rightarrow$-introduction\pause\vspace{0.14cm}\newline\pause
$(\psi \wedge \phi)$ est la conjonction de deux propositions, $\psi$ et $\phi$. On utilise donc la $\wedge$-introduction, qui impose de les prouver toutes les deux en utilisant le même jeu d'hypothèses\pause\vspace{0.14cm}\newline\pause
Dans la branche gauche (resp. droite), on veut prouver $\psi$ (resp. $\phi$). Or, on a comme hypothèse `$\phi \wedge \psi$`, qui permet de prouver directement $\psi$ (resp. $\phi$) d'après la $\wedge$-élimination. On fait donc \textit{apparaître} cette hypothèse en utilisant l'axiome


%Vous pouvez remarquer que les prémisses sont alignées au lieu d'être empilées (comme dans Port-Royal), on va très rapidement voir pourquoi ! 	
	
\end{frame}
%----------------------------------------


\begin{frame}
	\titre{Déduction naturelle}
	
	Autres règles : 
	 \begin{prooftree}
\AxiomC{$\Gamma \vdash \phi$}
%\UnaryInfC{B}
\UnaryInfC{$\Gamma \vdash (\phi \vee \psi)$}
\end{prooftree}

\begin{prooftree}
\AxiomC{$\Gamma \vdash \phi$}
%\UnaryInfC{B}
\UnaryInfC{$\Gamma \vdash (\psi \vee \phi)$}
\end{prooftree}


\pause

`Si on a une preuve de $\phi$ à partir de $\Gamma$, alors on a une preuve de $(\phi \vee \psi)$ (resp. $(\psi \vee \phi)$) à partir de $\Gamma$`. C'est les règles d'\textbf{introduction du $\vee$} (ou $\vee$-élimination)\pause\newline

Intuitivement, cette règle consiste à `brouiller les pistes`. `Si je sais que sous mes hypothèses machin est vrai, alors je sais que parmi machin et truc y aura au moins un de vrai, quel que soit truc`
	
\end{frame}

%----------------------------------------
\begin{frame}
	\titre{Déduction naturelle}
	
	On a notamment vu la $\wedge$-introduction et élimination. Vu qu'on vient de regarder la $\vee$-introduction, vous devez vous douter qu'on va avoir \pause la $\vee$-élimination\pause\newline
	
	 \begin{prooftree}
\AxiomC{$\Gamma \vdash (\phi \vee \psi)$}
\AxiomC{$\Gamma,\phi \vdash \omega$}
\AxiomC{$\Gamma,\psi \vdash \omega$}
%\UnaryInfC{B}
\TrinaryInfC{$\Gamma \vdash\omega$}
\end{prooftree}


\pause

Celle-là est un peu ésotérique, mais en fait assez normale : \pause si on peut prouver qu'on a soit $\phi$, soit $\psi$ (soit les deux), et que dans les deux cas on a $\omega$, alors on a forcément ce dernier

\end{frame}

%----------------------------------------
\begin{frame}
	\titre{Déduction naturelle}
	
On a aussi vu plus haut la $\rightarrow$-introduction, il nous manque donc la $\rightarrow$-elimination :
	
	 \begin{prooftree}
\AxiomC{$\Gamma \vdash (\phi \rightarrow \psi)$}
\AxiomC{$\Gamma \vdash \phi$}%\UnaryInfC{B}
\BinaryInfC{$\Gamma \vdash\psi$}
\end{prooftree}


\pause

Celle-là, qu'on appelle aussi \textbf{modus ponens} (quand on est un peu snob) correspond à un raisonnement de type `Si il pleut la route est mouillée, et il pleut, donc la route est mouillée`.

\end{frame}

%----------------------------------------

\begin{frame}
	%\titre{Déduction naturelle}
Autre preuve : 	

\only<1>{
\scalebox{.48}{}
\begin{scprooftree}{0.48}

\AxiomC{$((\phi \rightarrow \psi) \wedge (\psi \rightarrow \omega)),\phi \vdash ((\phi \rightarrow \psi) \wedge (\psi \rightarrow \omega))$}
\UnaryInfC{$((\phi \rightarrow \psi) \wedge (\psi \rightarrow \omega)),\phi \vdash (\psi \rightarrow \omega)$}
\AxiomC{$((\phi \rightarrow \psi) \wedge (\psi \rightarrow \omega)),\phi \vdash ((\phi \rightarrow \psi) \wedge (\psi \rightarrow \omega)) $}
\UnaryInfC{$((\phi \rightarrow \psi) \wedge (\psi \rightarrow \omega)),\phi \vdash (\phi \rightarrow \psi) $}
\AxiomC{$((\phi \rightarrow \psi) \wedge (\psi \rightarrow \omega)),\phi \vdash \phi $}
\BinaryInfC{$((\phi \rightarrow \psi) \wedge (\psi \rightarrow \omega)),\phi \vdash \psi $}
\BinaryInfC{$((\phi \rightarrow \psi) \wedge (\psi \rightarrow \omega)),\phi \vdash \omega$}
\UnaryInfC{$((\phi \rightarrow \psi) \wedge (\psi \rightarrow \omega)) \vdash (\phi \rightarrow \omega)$}
%\UnaryInfC{B}
\UnaryInfC{$ \vdash (((\phi \rightarrow \psi) \wedge (\psi \rightarrow \omega)) \rightarrow (\phi \rightarrow \omega))$}
\end{scprooftree}
}
\pause
\only<2->{

\scalebox{.8}{}
\begin{scprooftree}{0.8}
\AxiomC{$((\phi \rightarrow \psi) \wedge (\psi \rightarrow \omega)),\phi \vdash ((\phi \rightarrow \psi) \wedge (\psi \rightarrow \omega)) $}
\UnaryInfC{$((\phi \rightarrow \psi) \wedge (\psi \rightarrow \omega)),\phi \vdash (\phi \rightarrow \psi) $}
\AxiomC{$((\phi \rightarrow \psi) \wedge (\psi \rightarrow \omega)),\phi \vdash \phi $}
\BinaryInfC{$((\phi \rightarrow \psi) \wedge (\psi \rightarrow \omega)),\phi \vdash \psi $}
\end{scprooftree}

\scalebox{.8}{}
\begin{scprooftree}{0.8}

\AxiomC{$((\phi \rightarrow \psi) \wedge (\psi \rightarrow \omega)),\phi \vdash ((\phi \rightarrow \psi) \wedge (\psi \rightarrow \omega))$}
\UnaryInfC{$((\phi \rightarrow \psi) \wedge (\psi \rightarrow \omega)),\phi \vdash (\psi \rightarrow \omega)$}
\AxiomC{$((\phi \rightarrow \psi) \wedge (\psi \rightarrow \omega)),\phi \vdash \psi $}
\BinaryInfC{$((\phi \rightarrow \psi) \wedge (\psi \rightarrow \omega)),\phi \vdash \omega$}
\UnaryInfC{$((\phi \rightarrow \psi) \wedge (\psi \rightarrow \omega)) \vdash (\phi \rightarrow \omega)$}
%\UnaryInfC{B}
\UnaryInfC{$ \vdash (((\phi \rightarrow \psi) \wedge (\psi \rightarrow \omega)) \rightarrow (\phi \rightarrow \omega))$}
\end{scprooftree}
}
\pause 

Note : une preuve de $\vdash \psi$, ça veut dire que $\phi$ est vraie sans la moindre hypothèse. On parle alors de \textbf{théorème}

%Vous pouvez remarquer que les prémisses sont alignées au lieu d'être empilées (comme dans Port-Royal), on va très rapidement voir pourquoi ! 	
	
\end{frame}

%----------------------------------------

%----------------------------------------
\begin{frame}
	\titre{Déduction naturelle}
	
Courage, on a bientôt fini !
	
	 \begin{prooftree}
\AxiomC{$\Gamma \vdash \bot $}
\UnaryInfC{$\Gamma \vdash \phi$}
\end{prooftree}


\pause

Celle-là, c'est l'\textbf{élimination du faux}. Si l'ensemble d'hypothèses $\Gamma$ permet de prouver le faux, c'est qu'il est incohérent, du coup autant en déduire n'importe quoi tant qu'on y est

\end{frame}

%----------------------------------------
\begin{frame}
	\titre{Déduction naturelle}
	  
	Petit dernier
	
	 \begin{prooftree}
\AxiomC{$\Gamma, \neg \phi \vdash \bot $}
\UnaryInfC{$\Gamma \vdash \phi$}
\end{prooftree}


\pause

Celle-ci traduit \pause \textbf{le raisonnement par l'absurde} : si $\neg \phi$ fait \textit{bugger} l'ensemble d'hypothèses $\Gamma$, alors sa négation, cad $\phi$, est forcément vraie


\end{frame}

%----------------------------------------
\begin{frame}
	\titre{Déduction naturelle}
	  
	  Vous avez peut-être remarqué qu'un symbole de la logique prop. n'apparaît dans aucune des règles qu'on a vues \pause : en effet, on n'a pas croisé de $\leftrightarrow$\pause\newline
	  
	  En fait, les logiciens (qui sont à l'origine de la déduction naturelle) considèrent que $\phi \leftrightarrow \psi$ ne fait pas vraiment partie de logique prop., et qu'il ne s'agit que d'un raccourci pour $((\phi \rightarrow \psi) \wedge (\psi \rightarrow \phi))$ (ce qui se vérifie facilement avec des tables de vérité)\pause\newline
	  
	  Autre détail : $\neg \phi$ est également un \textit{raccourci} (on parle de `sucre syntaxique`) pour $(\phi \rightarrow \bot)$
	  

\end{frame}

%----------------------------------------


\begin{frame}
	\titre{Déduction naturelle}
	  
	La déduction naturelle a deux propriétés fort intéressantes vis-à-vis de la logique propositionnelle : \textbf{la correction} et \textbf{la complètude}\pause\newline
	
	La \textbf{correction}, ça veut dire que toute propriété prouvée dans e système qu'on vient de voir est vraie selon la logique prop. (cad que si on en fait la table de vérité, on obtient une tautologie). \pause\newline
	
	Bon, encore heureux quelque part. D'ailleurs la preuve est un peu trop technique pour qu'on la fasse, mais au fond putôt simple
	
	
	  

\end{frame}

%----------------------------------------

\begin{frame}
	\titre{Déduction naturelle}
	  
	La \textbf{complètude}, c'est que toute proposition qui est une tautologie (selon la logique prop) peut être prouvée en déduction naturelle \pause (preuve vraiment pas cool)\pause\newline
	
	\textbf{Correction} $\equiv$ (prouvable $\rightarrow$ vrai)\newline
	\textbf{Complètude} $\equiv$ (vrai $\rightarrow$ prouvable)\newline\pause
	
	Du coup, on peut construire sereinement l'entièreté des raisonnements valides \underline{en logique propositionnelle} avec même pas 10 règles, qu'on combine autant que besoin (cf la preuve homérique du théorème des 4 couleurs)\pause\newline
	
	Mais pourquoi s'embêter avec ça quand on peut tout vérifier avec une table de vérité ? \pause (sinon la beauté du geste)
		  

\end{frame}

%----------------------------------------

\begin{frame}
	\titre{Déduction naturelle}
	   
	  Vérification / recherche (automatisable) de preuve\newline\pause
	  
	  Correspondance preuves / programmes\newline\pause
	  
	  Extension à logique du premier ordre
	  
	  %Une raison très concrète : les tables de vérité ça ne marche que avec un nombre fini de propositions atomiques (sinon le tableau est lui-même infini et on n'est pas trop avancé). \pause En logique prop ça sera toujours le cas (la syntaxe autorise les propositions arbitrairement grandes mais pas infinies)\pause, mais comme on a commencé à le voir, c'est une logique trop faible pour exprimer tout un tas de choses\pause\newline
	  
	 % On va donc s'intéresser à la logique du premier ordre, extension fondamentale de la logique prop., dans laquelle les tables de vérité ne nous seront plus d'aucune aide. \pause Il y aura évidemment un équivalent (`faut bien définir formellement la notion de vérité), mais c'est encore pire que la déduction naturelle (qui sera elle-même adaptée pour conserver la complétude)\newline\pause
	  
	  %Pourquoi n'aura-t-on pas besoin de modifier la déduction naturelle pour conserver la correction ?

\end{frame}

%----------------------------------------
